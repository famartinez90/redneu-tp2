
\section{Introducción}


\textit{En este documento se realizan las actividades propuestas en el TP 2, actividades relacionadas con la implementación de dos problemas de clasificación.
El primero de ellos relacionado a la reducción de dimensionalidad de un set de datos. El segundo, confección de mapas autoorganizados. El objetivo del
trabajo es desarrollar redes neuronales que propongan soluciones a ambos.}

\subsection{Introducción al problema}
Las redes neuronales son modelos computacionales, en los que se intenta emular el funcionamiento fisiológico de un conjunto de neuronas biológicas, interconectadas, con el fin de lograr predicciones a partir de un conjunto de datos similares, presentados previamente. Para ello se modelan, en cada unidad de procesamiento, características que tienen que ver con las condiciones de propagación de señales electroquímicas. Estas condiciones se describen y modelan a partir de observaciones de sobre cómo es transmitida información entre una neurona y otra (o sobre si), y sobre como se encuentran interconectadas.

La suma de las interacciones entre estas unidades modeladas en una topología dada, genera propiedades emergentes que permiten  resolver cierto tipos de  problemas (en el caso de este trabajo, problemas de clasificación de elementos). 
Para intentar resolver estos problemas utilizando redes neuronales, es necesario recurrir a diversas técnicas para el ajuste de las variables de la red, y en muchos casos se requiere un paso de preprocesamiento de los datos.

En ambos casos utilizamos redes de aprendizaje no supervisado. Para el primer problema, la técnica utilizada fue aprendizaje hebbiano. Para el segundo, redes autoorganizadas. 

Los valores de entrada fueron clasificaciones de empresas brasileras en base a 850 atributos, originalmente derivados de una descripción en palabras. La idea final del trabajo fue poder predecir estas clasificaciones con redes de aprendizaje no supervisado y poder entrenarlas para lograrlo de la manera más precisa posible.

\subsection{Entrega}
\subsection{Requerimientos}
\begin{itemize}
\item Intérprete python 2.7.
\item Librerías estandar, librería \textit{matplotlib} y \textit{numpy}.
\item Archivo CSV con uno de los dos formatos propuestos en el TP. 
\end{itemize}

\subsection{Modo de uso y opciones}
Para usar este programa, se deben ejecutar el archivo \textbf{script.py}, el cual contiene todas las opciones de ejecución. Para ello, tipear por consola :

\texttt{\$python script.py N args}

donde \texttt{N} es el número de ejercicio y \texttt{args} son los argumentos optativos. Es obligatorio proveer el número de ejercicio, ya que se parsean los inputs de forma distinta. A su vez, cambian los métodos de cálculo de eficacia de la red.\\

Las opciones disponibles son:

\subsubsection{Opciones}

% \textbf{-file}: Filepath del dataset que se desee utilizar para entrenar o predecir resultados. Si no se lo provee, por default el programa buscará el archivo $tp1_ej1_training.csv$ o $tp1_ej2_training.csv$ en la carpeta donde se esté ejecutando, dependiendo del número de ejercicio pasado anteriormente.

% \textbf{-ep}: Cantidad de épocas por default, 500.

% \textbf{-eta}: Tasa de aprendizaje, por default $\eta$ = 0.05

% \textbf{-capas}: Capas ocultas de la red, cada número separado por una coma representa una capa y cada magnitud de la capa representa la cantidad de neuronas de esa capa, por default = '10,10', o sea, dos capas de 10 neuronas cada una.

% \textbf{-tr}:  Cantidad en \% del total de datos utilizado para entrenar a la red, por default = 70.

% \textbf{-te}: Cantidad en \% del total de datos utilizado para testing, por default = 20

% \textbf{-val}: Cantidad en \% del total de datos utilizados como validación, por default = 10

% \textbf{-tambatch}: Tamaño del batch de aprendizaje a utilizar. Por default el valor es $1$, es decir entrenamiento estocástico.

% \textbf{-mo}: Magnitud del momentum a utilizar. Default es 0.

% \textbf{-fa}: Función de activación, puede ser \textit{tangente o logística}, por default es \textit{tangente}.

% \textbf{-dp}: Distribución de inicialización de pesos a utilizar, puede ser \textit{normal} o \textit{uniforme}, por default se usa \textit{normal}.

% \textbf{-rda}: Permite cargar una red entrenada desde un archivo con formato \textit{JSON}. Se debe proveer el filepath del archivo JSON. En caso de no proveer este parámetro, se generará una red nueva, entrenándola con el dataset seleccionado.

% \textbf{-rha}: Permite almacenar una red entrenada a un archivo con \textit{JSON}. Se debe proveer el filepath destino del archivo JSON.

% \textbf{-estop}: Utilizado para activar early stopping. El argumento es el treshold a considerar. Default = 0.

% \textbf{-adap}: Define si se utiliza o no parámetros adaptativos. Valores = 0 o 1. Default = 0.

\subsection{Archivos}
Para la resolución del trabajo, fue necesario desarrollar en primer lugar un parser de los datos de entrada. El código de las funciones de parsing esta en \textbf{script.py}. Allí se encuentran la inicialización del programa, junto con el parseo de los parámetros, su transformación a media 0 y el llamado a las funciones y métodos de la red.\\

El código propiamente de la red se encuentra en \textbf{network.py}. Allí están todas las funciones de los procesos de inicialización, computación de pesos y resultados.\\

% El parsing y definición de parámetros está en \textbf{parameters.py}. Las funciones encargadas de la normalización de los datos se encuentra en \textbf{normalizer.py}. Las funciones encargadas de codificar y decodificar redes entrenadas en formato json se hallan en \textbf{encoder.py}\\

% \subsection{Otros scripts}

% Se proveen a su vez los scripts utilizados para los experimentos, en caso de desear ejecutarlos. En la sección Resultados comentaremos donde se encuentra el código utilizado para cada uno.
