
\section{Introducción}


\textit{En este documento se realizan las actividades propuestas en el TP 2, actividades relacionadas con la implementación de dos problemas de clasificación.
El primero de ellos relacionado a la reducción de dimensionalidad de un set de datos. El segundo, confección de mapas autoorganizados. El objetivo del
trabajo es desarrollar redes neuronales que propongan soluciones a ambos.}

\subsection{Introducción al problema}
Las redes neuronales son modelos computacionales, en los que se intenta emular el funcionamiento fisiológico de un conjunto de neuronas biológicas, interconectadas, con el fin de lograr predicciones a partir de un conjunto de datos similares, presentados previamente. Para ello se modelan, en cada unidad de procesamiento, características que tienen que ver con las condiciones de propagación de señales electroquímicas. Estas condiciones se describen y modelan a partir de observaciones de sobre cómo es transmitida información entre una neurona y otra (o sobre si), y sobre como se encuentran interconectadas.

La suma de las interacciones entre estas unidades modeladas en una topología dada, genera propiedades emergentes que permiten  resolver cierto tipos de  problemas (en el caso de este trabajo, problemas de clasificación de elementos). 
Para intentar resolver estos problemas utilizando redes neuronales, es necesario recurrir a diversas técnicas para el ajuste de las variables de la red, y en muchos casos se requiere un paso de preprocesamiento de los datos.

En ambos casos utilizamos redes de aprendizaje no supervisado. Para el primer problema, la técnica utilizada fue aprendizaje hebbiano. Para el segundo, redes autoorganizadas. 

Los valores de entrada fueron clasificaciones de empresas brasileras en base a 850 atributos, originalmente derivados de una descripción en palabras. La idea final del trabajo fue poder predecir estas clasificaciones con redes de aprendizaje no supervisado y poder entrenarlas para lograrlo de la manera más precisa posible.

\subsection{Entrega}
\subsection{Requerimientos}
\begin{itemize}
\item Intérprete python 2.7.
\item Librerías estandar, librería \textit{matplotlib} y \textit{numpy}.
\item Archivo CSV con dataset utilizando formato descripto en el TP. 
\end{itemize}

\subsection{Modo de uso y opciones}

Las soluciones para el ejercicio 1 y el 2 se encuentran en archivos diferentes.\\

Para probar la solución del ejercicio 1, se debe correr el archivo \textbf{script.py}, el cual contiene todas las opciones de ejecución. Para ello, tipear por consola :

\texttt{\$python script.py args}

donde \texttt{args} son los argumentos optativos.\\

Para ejecutar el código de la solución del ejercicio 2, se deben cargar el archivo \textbf{script\_map.py}, el cual contiene las mismas de ejecución. Para ello, ingresar por consola :

\texttt{\$python script\_map.py args}

donde \texttt{args} son los mismos argumentos optativos descriptos a continuación.\\

Las opciones disponibles son:

\subsubsection{Opciones}

\textbf{-file}: Filepath del dataset que se desee utilizar para entrenar o predecir resultados. Si no se lo provee, por default el programa buscará el archivo $tp2\_training\_dataset.csv$ en la carpeta donde se esté ejecutando. Se puede proveer otro dataset que respete exactamente el mismo formato que el default.

\textbf{-ep}: Cantidad de épocas por default, 200.

\textbf{-eta}: Tasa de aprendizaje, por default $\eta$ = 0.01

\textbf{-r}: Regla de aprendizaje a utilizar, valores posibles: ``oja'' o ``sanger''. Sólo válido para ejercicio 1.

\textbf{-dim}: Cantidad de dimensiones de los documentos a procesar. Esta dimensionalidad afecta a la salida, si se utiliza para el ejercicio 1 o a la entrada, si se utiliza para el ejercicio
2. El valor por defecto es 3.

\textbf{-rda}: Permite cargar una red entrenada desde un archivo con formato JSON. Se debe proveer el filepath del archivo JSON. En caso de no proveer este parámetro, se generará una red nueva, entrenándola con el dataset seleccionado.

\textbf{-rha}: Permite almacenar una red entrenada a un archivo con JSON. Se debe proveer el filepath destino del archivo JSON.

\textbf{-red\_ej1}: Este argumento permite proveer a la solución del ejercicio 2 una red entrenada del ejercicio 1 para calcular componentes principales y procesar el mapa en base a estas coordenadas. Sólo válido para ejercicio 2. 

\subsection{Archivos}

El código está dividido en varios archivos.\\
Para el ejercicio 1, podemos encontrar las clases y funciones propias de la red neuronal en \textbf{network.py}. El script que carga los documentos, los preprocesa, decide el tamaño de entrada y se encarga de graficar los resultados se encuentra en \textbf{script.py}.\\

En el ejercicio 2, podemos encontrar el código de la red neuronal en \textbf{self\_organized\_map.py}. El script hace las llamadas a la red y a todos los elementos relacionados se encuentra en 
\textbf{script\_map.py}.\\

Otros archivos importantes son \textbf{encoder.py} que tiene las funciones encargadas del guardado y cargado de redes a formato json para almacenar soluciones entrenadas. En \textbf{parameters.py} se encuentran las funciones encargadas de procesar los parámetros de entrada de los scripts de ejecución.

Incluímos dos redes entrenadas para el ejercicio 1 en formato JSON. Ambas entrenadas con los parámetros óptimos obtenidos de las conclusiones de los experimentos. Son las utilizadas para la generación de las coordenadas de entrada para el ejercicio 2. Se encuentran en \textbf{sanger200\_dim3.json} y \textbf{sanger200\_dim9.json}.
